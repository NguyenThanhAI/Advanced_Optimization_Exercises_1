\documentclass[14pt, a4paper]{article}
\usepackage{minitoc}
\usepackage[left=3.00cm, right=2.5cm, top=2.00cm, bottom=2.00cm]{geometry}
\usepackage{amsmath}
\usepackage{amssymb}
\usepackage{amsthm}
\usepackage{mathtools}
\usepackage{graphicx}
\usepackage{algpseudocode}
\usepackage{algorithm}
\usepackage{blindtext}
\usepackage{setspace}
\usepackage[utf8]{inputenc}
\usepackage[utf8]{vietnam}
\usepackage[center]{caption}
\usepackage[shortlabels]{enumitem}
\usepackage{fancyhdr} % header, footer
\usepackage{hyperref} % loại bỏ border với mục lục và công thức
\usepackage[nonumberlist, nopostdot, nogroupskip]{glossaries}
\usepackage{glossary-superragged}
\setglossarystyle{superraggedheaderborder}
\pagestyle{fancy}
%\usepackage[style=numeric,sortcites]{biblatex}
%\addbibresource{ref.bib}
%\usepackage[numbers]{natbib}
\usepackage{indentfirst}
\usepackage[natbib,backend=biber,style=ieee, sorting=ynt]{biblatex}
\bibliography{ref.bib}

\graphicspath{{./figures/}}

%\renewbibmacro*{cite}{%
%  \printtext[bibhyperref]{%
%    \printfield{prefixnumber}%
%    \printfield{labelnumber}%
%    \ifbool{bbx:subentry}%
%      {\printfield{entrysetcount}}%
%   \ifnumequal{\value{citecount}}{\value{citetotal}-1}%
%       {\gdef\multicitedelim{\addspace\bibstring{and}\space}}%
%       {\gdef\multicitedelim{\addcomma\space}}%
%    }%
%}

%\makenoidxglossaries
%
%% Danh mục thuật ngữ
%\newglossaryentry{GMRES}
%{
%	name={GMRES},
%	description={Generalized Minimal Residual Method}
%}
%
%\newglossaryentry{MINRES}
%{
%	name={MINRES},
%	description={Minimal Residual Method}
%}
%
%\hypersetup{
%    colorlinks=false,
%    pdfborder={0 0 0},
%}
%
%\title{Tiểu luận phương pháp số cho đại số tuyến tính}
%
%\author{Nguyễn Chí Thanh}
%
%%\date{24-04-2022}
\fancyhf{}
\rhead{\textbf{Môn học: Tối ưu hóa nâng cao}}
\lhead{\textbf{GVHD: TS. Hoàng Nam Dũng}}
\rfoot{\thepage}
\lfoot{\textbf{Học viên thực hiện: Nguyễn Chí Thanh - Nguyễn Đức Thịnh}}
\renewcommand{\headrulewidth}{0.4pt}
\renewcommand{\footrulewidth}{0.4pt}
%
%\numberwithin{equation}{section}
%\numberwithin{algorithm}{section}
%\numberwithin{figure}{section}
%
%\setlength{\parindent}{0.5cm}
%
%\setcounter{secnumdepth}{3} % Cho phép subsubsection trong report
%\setcounter{tocdepth}{3} % Chèn subsubsection vào bảng mục lục

%\newtheorem{dl}{Định lý}
%\newtheorem{md}{Mệnh đề}
%\newtheorem{bd}{Bổ đề}
%\newtheorem{dn}{Định nghĩa}
%\newtheorem{hq}{Hệ quả}

%\newtheorem{baitap}{Bài tập}
%\newtheorem*{loigiai}{Lời giải}

%\numberwithin{dl}{section}
%\numberwithin{md}{section}
%\numberwithin{bd}{section}
%\numberwithin{dn}{section}
%\numberwithin{hq}{section}

\newtheoremstyle{sltheorem}
{}                % Space above
{}                % Space below
{\normalfont}        % Theorem body font % (default is "\upshape")
{}                % Indent amount
{\bfseries}       % Theorem head font % (default is \mdseries)
{.}               % Punctuation after theorem head % default: no punctuation
{ }               % Space after theorem head
{}                % Theorem head spec
\theoremstyle{sltheorem}
\newtheorem{baitap}{Bài tập}
\newtheorem*{loigiai}{Lời giải}

\doublespacing

\begin{document}

    \nocite{*}

    \begin{baitap}
        Hãy chứng minh các hàm sau là hàm lồi

        \begin{enumerate}[wide, labelwidth=!, labelindent=0pt,label=\textbf{\arabic*}.]
            \item Ridge Regression
            \begin{equation*}
                f(x)=\lVert Ax - b \rVert_2^2 + \lambda \lVert x \rVert_2^2
            \end{equation*}
            \item Lasso Regression
            \begin{equation*}
                f(x)=\lVert Ax - b \rVert_2^2 + \lambda \lVert x \rVert_1
            \end{equation*}
            \item Logistic Regression
            \begin{equation*}
                f(w)=\dfrac{1}{n}\sum_{i=1}^n \Big( y_i x_i^T w + \ln(1 + e^{x_i^T w}) \Big)
            \end{equation*}
            với $(x_i, y_i) \in \mathbb{R}^{k+1},i=1,2,\dots,n$ là cặp dữ liệu đầu vào và nhãn tương ứng và $w \in \mathbb{R}^k$
        \end{enumerate}
    \end{baitap}

    \begin{loigiai}
        
    \end{loigiai}

    \begin{baitap}
        Cho hàm số $f(x, y) = x^4 + x^2 - 6xy + 3y^2$. Hãy tìm các điểm cực trị địa phương của $f$, hãy chỉ rõ đó là cực đại hay cực tiểu địa phương
    \end{baitap}

    \begin{loigiai}
    \end{loigiai}

    \begin{baitap}
        Cho $A \in \mathbb{R}^{m \times n}$ với $m > n$ và $\mathrm{rank}(A)=n$, $b \in \mathbb{R}^{m}$. Hãy giải bài toán tối ưu sau:
        \begin{equation*}
            \min_x \lVert Ax - b \rVert_2^2
        \end{equation*}
    \end{baitap}

    \begin{loigiai}
    \end{loigiai}

    \begin{baitap}
        Xét hàm số Rosenbrock
        \begin{equation*}
            f(x)=100(x_2 - x_1^2)^2 + (1-x_1)^2
        \end{equation*}
        \begin{enumerate}[wide, labelwidth=!, labelindent=0pt,label=\textbf{\arabic*}.]
            \item $f$ có phải là hàm lồi không? Tại sao?
            \item Hãy tìm tất cả các điểm cực tiểu địa phương của $f$. Hàm $f$ có điểm cực tiểu toàn cục không? Nếu có hãy chỉ ra điểm đó là điểm nào?
        \end{enumerate}
    \end{baitap}

    \begin{loigiai}

    \end{loigiai}

    \newpage
    \printbibliography[title={TÀI LIỆU THAM KHẢO}]
\end{document}