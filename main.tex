\documentclass[14pt, a4paper]{article}
\usepackage{minitoc}
\usepackage[left=3.00cm, right=2.5cm, top=2.00cm, bottom=2.00cm]{geometry}
\usepackage{amsmath}
\usepackage{amssymb}
\usepackage{amsthm}
\usepackage{mathtools}
\usepackage{graphicx}
\usepackage{algpseudocode}
\usepackage{algorithm}
\usepackage{blindtext}
\usepackage{setspace}
\usepackage[utf8]{inputenc}
\usepackage[utf8]{vietnam}
\usepackage[center]{caption}
\usepackage[shortlabels]{enumitem}
\usepackage{fancyhdr} % header, footer
\usepackage{hyperref} % loại bỏ border với mục lục và công thức
\usepackage[nonumberlist, nopostdot, nogroupskip]{glossaries}
\usepackage{glossary-superragged}
\setglossarystyle{superraggedheaderborder}
\pagestyle{fancy}
%\usepackage[style=numeric,sortcites]{biblatex}
%\addbibresource{ref.bib}
%\usepackage[numbers]{natbib}
\usepackage{indentfirst}
\usepackage[natbib,backend=biber,style=ieee, sorting=ynt]{biblatex}
\bibliography{ref.bib}

\graphicspath{{./figures/}}

%\renewbibmacro*{cite}{%
%  \printtext[bibhyperref]{%
%    \printfield{prefixnumber}%
%    \printfield{labelnumber}%
%    \ifbool{bbx:subentry}%
%      {\printfield{entrysetcount}}%
%   \ifnumequal{\value{citecount}}{\value{citetotal}-1}%
%       {\gdef\multicitedelim{\addspace\bibstring{and}\space}}%
%       {\gdef\multicitedelim{\addcomma\space}}%
%    }%
%}

%\makenoidxglossaries
%
%% Danh mục thuật ngữ
%\newglossaryentry{GMRES}
%{
%	name={GMRES},
%	description={Generalized Minimal Residual Method}
%}
%
%\newglossaryentry{MINRES}
%{
%	name={MINRES},
%	description={Minimal Residual Method}
%}
%
%\hypersetup{
%    colorlinks=false,
%    pdfborder={0 0 0},
%}
%
%\title{Tiểu luận phương pháp số cho đại số tuyến tính}
%
%\author{Nguyễn Chí Thanh}
%
%%\date{24-04-2022}
\fancyhf{}
\rhead{\textbf{Môn học: Tối ưu hóa nâng cao}}
\lhead{\textbf{GVHD: TS. Hoàng Nam Dũng}}
\rfoot{\thepage}
\lfoot{\textbf{Học viên thực hiện: Nguyễn Chí Thanh - Nguyễn Đức Thịnh}}
\renewcommand{\headrulewidth}{0.4pt}
\renewcommand{\footrulewidth}{0.4pt}
%
%\numberwithin{equation}{section}
%\numberwithin{algorithm}{section}
%\numberwithin{figure}{section}
%
%\setlength{\parindent}{0.5cm}
%
%\setcounter{secnumdepth}{3} % Cho phép subsubsection trong report
%\setcounter{tocdepth}{3} % Chèn subsubsection vào bảng mục lục

%\newtheorem{dl}{Định lý}
%\newtheorem{md}{Mệnh đề}
%\newtheorem{bd}{Bổ đề}
%\newtheorem{dn}{Định nghĩa}
%\newtheorem{hq}{Hệ quả}

%\newtheorem{baitap}{Bài tập}
%\newtheorem*{loigiai}{Lời giải}

%\numberwithin{dl}{section}
%\numberwithin{md}{section}
%\numberwithin{bd}{section}
%\numberwithin{dn}{section}
%\numberwithin{hq}{section}

\setlength{\parindent}{0cm}

\newtheoremstyle{sltheorem}
{}                % Space above
{}                % Space below
{\normalfont}        % Theorem body font % (default is "\upshape")
{}                % Indent amount
{\bfseries}       % Theorem head font % (default is \mdseries)
{.}               % Punctuation after theorem head % default: no punctuation
{ }               % Space after theorem head
{}                % Theorem head spec
\theoremstyle{sltheorem}
\newtheorem{baitap}{Bài tập}
\newtheoremstyle{soltheorem}
{}                % Space above
{}                % Space below
{\normalfont}        % Theorem body font % (default is "\upshape")
{}                % Indent amount
{\bfseries}       % Theorem head font % (default is \mdseries)
{.}               % Punctuation after theorem head % default: no punctuation
{\newline}               % Space after theorem head
{}                % Theorem head spec
\theoremstyle{soltheorem}
\newtheorem*{loigiai}{Lời giải}

\onehalfspacing

\begin{document}

    \begin{titlepage}

        \newcommand{\HRule}{\rule{\linewidth}{0.5mm}} % Defines a new command for the horizontal lines, change thickness here

        \center % Center everything on the page

        %----------------------------------------------------------------------------------------
        %	HEADING SECTIONS
        %----------------------------------------------------------------------------------------
        \textsc{\LARGE Đại học Quốc Gia Hà Nội}\\[0.5cm]
        \textsc{\LARGE Đại học Khoa học tự nhiên}\\[0.5cm] % Name of your university/college
        \textsc{\LARGE Khoa Toán - Cơ - Tin học}\\[0.5cm]

        \includegraphics[scale=0.2]{HUS-logo.jpg}\\[0.5cm]

        \textsc{\Large Chuyên ngành: Khoa học dữ liệu}\\[0.5cm] % Major heading such as course name


        %----------------------------------------------------------------------------------------
        %	TITLE SECTION
        %----------------------------------------------------------------------------------------

        \HRule \\[0.4cm]
        { \huge \bfseries Bài tập môn học}\\[0.4cm] % Title of your document
        \HRule \\[1.5cm]

        \textsc{\Large Môn học: Tối ưu hóa nâng cao}\\[1.5cm] % Minor heading such as course title


        \textsc{\Large Bài tập 1}\\[1.5cm]


        %----------------------------------------------------------------------------------------
        %	AUTHOR SECTION
        %----------------------------------------------------------------------------------------
        \begin{minipage}{0.4\textwidth}
            \begin{flushleft} \Large
            \emph{Giảng viên hướng dẫn:} \\
            TS. Hoàng Nam Dũng % Supervisor's Name
            \end{flushleft}
        \end{minipage}\\[1cm]

        \begin{minipage}{0.4\textwidth}
        \begin{flushleft} \Large
        \emph{Nhóm học viên thực hiện:}\\
        Nguyễn Chí Thanh \\
        MSHV: 21007925 \\ % Your name
        Nguyễn Đức Thịnh \\
        MSHV: 21007923 \\
        Lớp: Khoa học dữ liệu - K4
        \end{flushleft}
        \end{minipage}


        % If you don't want a supervisor, uncomment the two lines below and remove the section above
        %\Large \emph{Author:}\\
        %John \textsc{Smith}\\[3cm] % Your name

        %----------------------------------------------------------------------------------------
        %	DATE SECTION
        %----------------------------------------------------------------------------------------

        % I don't want day because it is English
        % {\large \today}\\[2cm] % Date, change the \today to a set date if you want to be precise

        %----------------------------------------------------------------------------------------
        %	LOGO SECTION
        %----------------------------------------------------------------------------------------

        %\includegraphics{logo/rsz_3logo-khtn.png}\\[1cm] % Include a department/university logo - this will require the graphicx package

        %----------------------------------------------------------------------------------------

        \vfill % Fill the rest of the page with whitespace

    \end{titlepage}

    \nocite{*}

    \newpage

    \begin{baitap}
        Hãy chứng minh các hàm sau là hàm lồi

        \begin{enumerate}[wide, labelwidth=!, labelindent=0pt,label=\textbf{\arabic*}.]
            \item Ridge Regression
            \begin{equation*}
                f(x)=\lVert Ax - b \rVert_2^2 + \lambda \lVert x \rVert_2^2
            \end{equation*}
            \item Lasso Regression
            \begin{equation*}
                f(x)=\lVert Ax - b \rVert_2^2 + \lambda \lVert x \rVert_1
            \end{equation*}
            \item Logistic Regression
            \begin{equation*}
                f(w)=\dfrac{1}{n}\sum_{i=1}^n \Big( -y_i x_i^T w + \ln(1 + e^{x_i^T w}) \Big)
            \end{equation*}
            với $(x_i, y_i) \in \mathbb{R}^{k+1},i=1,2,\dots,n$, là cặp dữ liệu đầu vào và nhãn tương ứng và $w \in \mathbb{R}^k$
        \end{enumerate}
    \end{baitap}

    \begin{loigiai}

        Hãy chứng minh các hàm sau là hàm lồi
        \begin{enumerate} [wide, labelwidth=!, labelindent=0pt,label=\textbf{\arabic*}.]
            \item Ridge Regression
            \begin{equation*}
                f(x)=\lVert Ax - b \rVert_2^2 + \lambda \lVert x \rVert_2^2
            \end{equation*}

            Ta đặt $A \in \mathbb{R}^{m \times n}, b \in \mathbb{R}^{m}, x \in \mathbb{R}^{n}$, ta có $\mathrm{dom}(f)=\mathbb{R}^{n}$ là tập lồi

            \begin{equation*}
                \begin{aligned}
                    f(x) &= (Ax - b)^T (Ax - b) + \lambda x^T x \\
                    &=x^T A^T A x - x^TA^Tb - b^T Ax + b^T b + \lambda x^T x
                \end{aligned}
            \end{equation*}
            Do $x^TA^Tb$ và $b^TAx$ là hai số vô hướng và $(x^TA^Tb)^T=b^TAx$ nên $x^TA^Tb=b^TAx$ (chuyển vị của một số vô hướng bằng chính nó) nên:

            \begin{equation*}
                \begin{aligned}
                    f(x) &= x^T A^T A x - x^TA^Tb - b^T Ax + b^T b + \lambda x^T x\\
                    &=x^T A^T A x-2x^TA^Tb+b^Tb + \lambda x^T x
                \end{aligned}
            \end{equation*}
            Ta tính Gradient của $f(x)$ theo $x$:

            \begin{equation*}
                \nabla f(x) =\Big\lbrack A^TA + (A^TA)^T \Big\rbrack x - 2A^Tb + 2\lambda x = 2A^T A x - 2 A^T b + 2\lambda x
            \end{equation*}

            Ta tính ma trận Hessian của $f(x)$ theo $x$:

            \begin{equation*}
                \nabla^2 f(x) = 2A^T A + 2 \lambda I
            \end{equation*}

            Ta xét:

            \begin{equation*}
                p^T \nabla^2 f(x) p = 2 p^T A^T A p + 2\lambda p^Tp = 2 (Ap)^TAp + 2\lambda p^Tp \geq 0 \thickspace\forall \thickspace p \in \mathbb{R}^{n}
            \end{equation*}

            nên:

            \begin{equation*}
                \nabla^2 f(x) \succeq 0
            \end{equation*}

            Vì $\mathrm{dom}(f)$ là tập lồi và $\nabla^2 f(x) \succeq 0$ nên hàm $f(x)$ là một hàm lồi.

            \item Lasso Regression
            \begin{equation*}
                f(x)=\lVert Ax - b \rVert_2^2 + \lambda \lVert x \rVert_1
            \end{equation*}

            Ta đặt $A \in \mathbb{R}^{m \times n}, b \in \mathbb{R}^{m}, x \in \mathbb{R}^{n}$, ta có $\mathrm{dom}(f)=\mathbb{R}^{n}$ là tập lồi,
            $\mathrm{dom}(\lVert Ax - b \rVert_2^2)=\mathbb{R}^n$ và $\mathrm{dom}(\lVert x \rVert_1)=\mathbb{R}^n$

            Ta chứng minh $\lVert Ax - b \rVert_2^2$ là hàm lồi:

            \begin{equation*}
                \begin{aligned}
                    \lVert Ax - b \rVert_2^2&=(Ax-b)^T(Ax-b)\\
                    &=x^TA^TAx - x^TA^Tb - b^TAx + b^Tb
                \end{aligned}
            \end{equation*}
            Do $x^TA^Tb$ và $b^TAx$ là hai số vô hướng và $(x^TA^Tb)^T=b^TAx$ nên $x^TA^Tb=b^TAx$ (chuyển vị của một số vô hướng bằng chính nó) nên:

            \begin{equation*}
                \lVert Ax - b \rVert_2^2=x^TA^TAx - 2x^TA^Tb + b^Tb
            \end{equation*}

            Ta tính Gradient của $\lVert Ax - b \rVert_2^2$ theo $x$:

            \begin{equation*}
                \nabla \lVert Ax - b \rVert_2^2=\Big \lbrack A^TA + (A^TA)^T \Big \rbrack x-2A^Tb=2A^TAx - 2A^Tb
            \end{equation*}

            Ta tính ma trận Hessian của $\lVert Ax - b \rVert_2^2$ theo $x$:
            \begin{equation*}
                \nabla^2 \lVert Ax - b \rVert_2^2=2A^TA
            \end{equation*}

            Ta xét:

            \begin{equation*}
                2p^TA^TAp=2(Ap)^TAp\geq0 \thickspace \forall p \in \mathbb{R}^{n} \Rightarrow 2A^TA \succeq 0
            \end{equation*}

            Mặt khác, $\mathrm{dom}(\lVert Ax - b \rVert_2^2)=\mathbb{R}^n$ là tập lồi. Vậy nên $\lVert Ax - b \rVert_2^2$ là hàm lồi

            Ta chứng minh $\lVert x \rVert_1$ là hàm lồi, với $\theta \in \lbrack 0, 1\rbrack, x,y \in \mathbb{R}^n$:

            \begin{equation*}
                \begin{aligned}
                    \lVert \theta x + (1-\theta)y \rVert_1 \leq \theta \lVert x \rVert_1 + (1-\theta)\lVert y \rVert_1 \thickspace \forall \thickspace \theta \in \lbrack 0, 1 \rbrack, x, y \in \mathbb{R}^n
                \end{aligned}
            \end{equation*}

            theo bất đẳng thức tam giác và $\mathrm{dom}(\lVert x \rVert_1)=\mathbb{R}^n$ là tập lồi nên $\lVert x \rVert_1$ là hàm lồi. 
            $\mathrm{dom}(f)$ là một tập lồi và tổng của hai hàm lồi là một hàm lồi nên $f(x)$ là một hàm lồi.

            \item Logistic Regression
            \begin{equation*}
                f(w)=\dfrac{1}{n}\sum_{i=1}^n \Big( -y_i x_i^T w + \ln(1 + e^{x_i^T w}) \Big)
            \end{equation*}
            với $(x_i, y_i) \in \mathbb{R}^{k+1},i=1,2,\dots,n$ là cặp dữ liệu đầu vào và nhãn tương ứng và $w \in \mathbb{R}^k$

            Ta có $\mathrm{dom}(f(w)=\mathbb{R}^{k}$, là một tập lồi. Ta tính Gradient của $f(w)$ theo $w$:

            \begin{equation*}
                \nabla f(w)=\dfrac{1}{n}\sum_{i=1}^n \Big( -y_i x_i + \dfrac{e^{x_i^Tw}}{1 + e^{x_i^T w}}x_i \Big)
            \end{equation*}

            Ta tính ma trận Hessian của $f(w)$ theo $w$:

            \begin{equation*}
                \nabla^2 f(x) = \dfrac{1}{n}\sum_{i=1}^n \dfrac{e^{x_i^Tw}(1 + e^{x_i^T w})-e^{2x_i^T w}}{(1 + e^{x_i^T w})^2}x_i x_i^T=\dfrac{1}{n}\sum_{i=1}^n \dfrac{e^{x_i^Tw}}{(1 + e^{x_i^T w})^2}x_i x_i^T
            \end{equation*}

            Ta xét số hạng thứ $i$:

            \begin{equation*}
                \dfrac{e^{x_i^Tw}}{(1 + e^{x_i^T w})^2}x_i x_i^T
            \end{equation*}

            Ta xét ma trận Hessian $\nabla^2 f(x)$:

            \begin{equation*}
                \dfrac{e^{x_i^Tw}}{(1 + e^{x_i^T w})^2}p^Tx_i x_i^T p=\Bigg(\sqrt{\dfrac{e^{x_i^Tw}}{(1 + e^{x_i^T w})^2}} x_i^Tp\Bigg)^T\Bigg(\sqrt{\dfrac{e^{x_i^Tw}}{(1 + e^{x_i^T w})^2}} x_i^Tp\Bigg)\geq 0 \thickspace \forall p \in \mathbb{R}^{k}
            \end{equation*}
            \begin{equation*}
                \begin{aligned}
                    \Rightarrow p^T \nabla^2 f(x)p&=\dfrac{1}{n}\sum_{i=1}^n \dfrac{e^{x_i^Tw}}{(1 + e^{x_i^T w})^2}p^Tx_i x_i^T\\
                    &=\dfrac{1}{n}\sum_{i=1}^n \Bigg(\sqrt{\dfrac{e^{x_i^Tw}}{(1 + e^{x_i^T w})^2}} x_i^Tp\Bigg)^T\Bigg(\sqrt{\dfrac{e^{x_i^Tw}}{(1 + e^{x_i^T w})^2}} x_i^Tp\Bigg)\geq 0 \thickspace \forall p \in \mathbb{R}^{k}
                \end{aligned}
            \end{equation*}

            Ta có $\mathrm{dom}(f(w))=\mathbb{R}^{k}$ là một tập lồi và $\nabla^2 f(x) \succeq 0$ nên $f(w)$ là hàm lồi
        \end{enumerate}
        

    \end{loigiai}

    \begin{baitap}
        Cho hàm số $f(x, y) = x^4 + x^2 - 6xy + 3y^2$. Hãy tìm các điểm cực trị địa phương của $f$, hãy chỉ rõ đó là cực đại hay cực tiểu địa phương
    \end{baitap}

    \begin{loigiai}
        Ta có $\mathrm{dom}(f)=\mathbb{R}^2$ là một tập lồi. Ta tính Gradient của $f(x, y)$ theo $x$ và $y$:

        \begin{equation*}
            \nabla f(x, y) = \begin{bmatrix} \dfrac{\partial f(x, y)}{\partial x} \\ \dfrac{\partial f(x, y)}{\partial y}\end{bmatrix} = \begin{bmatrix} 4x^3 + 2x - 6y \\ -6x + 6y \end{bmatrix}
        \end{equation*}
        Ta tính ma trận Hessian của $f(x, y)$ theo $x$ và $y$:

        \begin{equation*}
            \nabla^2 f(x,y)=\begin{bmatrix} \dfrac{\partial^2 f(x, y)}{\partial x^2} & \dfrac{\partial^2 f(x, y)}{\partial x\partial y} \\ \dfrac{\partial^2 f(x, y)}{\partial y \partial x } & \dfrac{\partial^2 f(x, y)}{\partial y^2 } \end{bmatrix}=\begin{bmatrix} 12x^2 + 2 & -6 \\ -6 & 6 \end{bmatrix}
        \end{equation*}

        Ta nhận thấy $\nabla^2 f(x, y)$ tồn tại và liên tục với mọi $(x, y) \in \mathbb{R}^{2}$

        Ta xét:

        \begin{equation*}
            \det(\begin{bmatrix} 12x^2 + 2  \end{bmatrix})>0 \thickspace \forall (x, y) \in \mathbb{R}^2
        \end{equation*}

        \begin{equation*}
            \det(\nabla^2 f(x,y))=\det\Bigg(\begin{bmatrix} 12x^2 + 2 & -6 \\ -6 & 6 \end{bmatrix} \Bigg)=6(12x^2 + 2) - 36 =72x^2 - 24
        \end{equation*}

        \begin{itemize}
            \item $\det\Big(\nabla^2 f(x,y)\Big) < 0$ với $x \in \Big(-\dfrac{1}{\sqrt{3}}; \dfrac{1}{\sqrt{3}}\Big)$, $\nabla^2 f(x,y)$ là ma trận xác định âm.
            \item $\det\Big(\nabla^2 f(x,y)\Big) > 0$ với $x \in \Big(-\infty;-\dfrac{1}{\sqrt{3}}\Big) \cup \Big(\dfrac{1}{\sqrt{3}}; +\infty\Big)$, $\nabla^2 f(x,y)$ là ma trận xác định dương.
        \end{itemize}

        Như vậy $\nabla^2 f(x,y)$ là ma trận không xác định dấu do $\nabla^2 f(x,y) \succ 0$ hoặc $\nabla^2 f(x,y) \prec 0$ phụ thuộc vào giá trị $x$.
        
        Để tìm các điểm dừng, ta giải phương trình $\nabla f(x, y) = 0$

        \begin{equation*}
            \nabla f(x, y) = \begin{bmatrix} 4x^3 + 2x - 6y \\ -6x + 6y \end{bmatrix} = \begin{bmatrix} 0 \\ 0 \end{bmatrix}
        \end{equation*}

        \begin{equation}
            \begin{aligned}
            &\Leftrightarrow \begin{cases} 4x^3 + 2x - 6y = 0 \\ x = y \end{cases} \\
            &\Leftrightarrow \begin{cases} 4x(x^2-1) = 0 \\ x = y \end{cases} \\
            &\Leftrightarrow \left [\begin{array}{l} x=y=0 \\ x=y=1 \\ x=y=-1 \end{array}\right.
            \end{aligned}
        \end{equation}

        Ta sẽ xét ma trận Hessian $\nabla^2 f(x,y)$ tại các điểm dừng nghiệm của phương trình $\nabla f(x, y) = 0$:

        \begin{itemize}
            \item $\nabla^2 f(x=0,y=0)=\begin{bmatrix} 2 & -6 \\ -6 & 6 \end{bmatrix}$. Ta có $\det(\begin{bmatrix}2\end{bmatrix})=2>0, \det{\Big(\begin{bmatrix} 2 & -6 \\ -6 & 6 \end{bmatrix}\Big)}=2\times6 - (-6)\times(-6)=-24<0$. $\nabla^2 f(x=0,y=0)$ là ma trận xác định âm. 
            Vì $\nabla^2 f(x, y)$ tồn tại và liên tục với mọi $(x, y) \in \mathbb{R}^2$ nên $\nabla^2 f(x, y)$ cũng tồn tại là liên tục trong một lân cận mở của $(x,y)=(0,0)$, $\nabla f(x=0,y=0)=0$ và $\nabla^2 f(x=0, y=0)$ là ma trận xác định âm nên $(x,y)=(0,0)$ là một điểm cực đại địa phương ngặt.
            \item $\nabla^2 f(x=1,y=1)=\begin{bmatrix} 14 & -6 \\ -6 & 6 \end{bmatrix}$. $\det{(\begin{bmatrix}14\end{bmatrix})}=14>0, \det{\Big(\begin{bmatrix} 14 & -6 \\ -6 & 6 \end{bmatrix}\Big)}=14\times6 - (-6)\times(-6)=48>0$.
            $\nabla^2 f(x=1,y=1)$ là ma trận xác định dương.
            Vì $\nabla^2 f(x, y)$ tồn tại và liên tục với mọi $(x, y) \in \mathbb{R}^2$ nên $\nabla^2 f(x, y)$ cũng tồn tại là liên tục trong một lân cận mở của $(x,y)=(1,1)$, $\nabla f(x=1,y=1)=0$ và $\nabla^2 f(x=1,y=1)$ là ma trận xác định dương nên $(x,y)=(1,1)$ là một cực tiểu địa phương ngặt.
            \item $\nabla^2 f(x=-1,y=-1)=\begin{bmatrix} 14 & -6 \\ -6 & 6 \end{bmatrix}$. $\det{(\begin{bmatrix}14\end{bmatrix})}=14>0, \det{\Big(\begin{bmatrix} 14 & -6 \\ -6 & 6 \end{bmatrix}\Big)}=14\times6 - (-6)\times(-6)=48>0$.
            $\nabla^2 f(x=-1,y=-1)$ là ma trận xác định dương.
            Vì $\nabla^2 f(x, y)$ tồn tại và liên tục với mọi $(x, y) \in \mathbb{R}^2$ nên $\nabla^2 f(x, y)$ cũng tồn tại là liên tục trong một lân cận mở của $(x,y)=(-1,-1)$, $\nabla f(x=-1,y=-1)=0$ và $\nabla^2 f(x=-1,y=-1)$ là ma trận xác định dương nên Vậy $x=y=-1$ là một cực tiểu địa phương.
        \end{itemize}
        Vậy ta có $(x,y)=(0,0)$ là một điểm cực đại địa phương và hai điểm $(x,y)=(1, 1)$ và $(x,y)=(-1,-1)$ là các điểm cực tiểu địa phương.
    \end{loigiai}

    \begin{baitap}
        Cho $A \in \mathbb{R}^{m \times n}$ với $m > n$ và $\mathrm{rank}(A)=n$, $b \in \mathbb{R}^{m}$. Hãy giải bài toán tối ưu sau:
        \begin{equation*}
            \min_x \lVert Ax - b \rVert_2^2
        \end{equation*}
    \end{baitap}

    \begin{loigiai}
        Ta có $\mathrm{dom}\Big( \lVert Ax-b \rVert_2^2\Big)=\mathbb{R}^{n}$ là một tập lồi. Ta sẽ chứng minh
        $\lVert Ax - b \rVert_2^2$ là hàm lồi. Ta xét:

        \begin{equation*}
            \begin{aligned}
            \lVert Ax - b \rVert_2^2&=(Ax-b)^T(Ax-b)\\
            &=x^TA^TAx - x^TA^Tb - b^TAx + b^Tb
            \end{aligned}
        \end{equation*}

        Do $x^TA^Tb$ và $b^TAx$ là hai số vô hướng và $(x^TA^Tb)^T=b^TAx$ nên $x^TA^Tb=b^TAx$ (chuyển vị của một số vô hướng bằng chính nó) nên:
        \begin{equation*}
            \lVert Ax - b \rVert_2^2=x^TA^TAx - 2x^TA^Tb + b^Tb
        \end{equation*}
        Ta nhận thấy $\lVert Ax - b \rVert_2^2$ là hàm khả vi tại mọi điểm $x \in \mathbb{R}^n$. Ta tính Gradient của $\lVert Ax - b \rVert_2^2$ theo $x$:

        \begin{equation*}
            \nabla \lVert Ax - b \rVert_2^2=\Big\lbrack A^TA + (A^TA)^T \Big\rbrack x-2A^Tb=2A^TAx - 2A^Tb
        \end{equation*}

        Ta tính ma trận Hessian của $\lVert Ax - b \rVert_2^2$ theo $x$:

        \begin{equation*}
            \nabla^2 \lVert Ax - b \rVert_2^2=2A^TA
        \end{equation*}

        Ta xét:

        \begin{equation*}
            2p^TA^TAp=2(Ap)^TAp\geq0 \thickspace \forall p \in \mathbb{R}^{n} \Rightarrow 2A^TA \succeq 0 \Rightarrow \nabla^2 \lVert Ax - b \rVert_2^2 \succeq 0
        \end{equation*}

        Ta có $\mathrm{dom}\Big( \lVert Ax-b \rVert_2^2\Big)=\mathbb{R}^{n}$ là một tập lồi và $\nabla^2 \lVert Ax - b \rVert_2^2 \succeq 0$. Vì vậy $\lVert Ax - b \rVert_2^2$ là hàm lồi. Mặt khác, $\lVert Ax - b \rVert_2^2$ khả vi tại mọi điểm $x \in \mathbb{R}^{n}$ nên mọi điểm dừng của hàm $\lVert Ax - b \rVert_2^2$ là cực tiểu toàn cục.

        Ta giải phương trình $\nabla \lVert Ax - b \rVert_2^2=0$:

        \begin{equation*}
            \nabla \lVert Ax - b \rVert_2^2 = 2 A^T A x - 2 A^T b = 0
        \end{equation*}

        Với $\mathrm{rank}(A)=n$, ta chứng minh $A^T A$ khả nghịch.

        Ta xét hệ phương trình:

        \begin{equation*}
            A^T A x = 0 \Rightarrow x^T A^T A x = 0 \Leftrightarrow (Ax)^T (Ax) = 0 \Leftrightarrow \lVert Ax \rVert_2^2 =0 \Leftrightarrow Ax = 0
        \end{equation*}

        Do $\mathrm{rank}(A)=n$ nên các cột của $A$ độc lập tuyến tính vậy để $Ax=0 \Leftrightarrow x = 0$
        Vậy $A^T A x = 0 \Leftrightarrow x = 0$, nên các cột của $A^T A$ độc lập tuyến tính, ta suy ra $A^T A$ khả nghịch.

        Vậy ta có nghiệm tối ưu toàn cục của bài toán là nghiệm của phương trình $\nabla \lVert Ax - b \rVert_2^2=0$:

        \begin{equation*}
            x^* = \underset{x}{\mathrm{argmin}}\lVert Ax - b \rVert_2^2=(A^TA)^{-1}A^T b
        \end{equation*}
    \end{loigiai}

    \begin{baitap}
        Xét hàm số Rosenbrock
        \begin{equation*}
            f(x)=100(x_2 - x_1^2)^2 + (1-x_1)^2
        \end{equation*}
        \begin{enumerate}[wide, labelwidth=!, labelindent=0pt,label=\textbf{\arabic*}.]
            \item $f$ có phải là hàm lồi không? Tại sao?
            \item Hãy tìm tất cả các điểm cực tiểu địa phương của $f$. Hàm $f$ có điểm cực tiểu toàn cục không? Nếu có hãy chỉ ra điểm đó là điểm nào?
        \end{enumerate}
    \end{baitap}

    \begin{loigiai}
        Xét hàm số Rosenbrock
        \begin{equation*}
            f(x)=100(x_2 - x_1^2)^2 + (1-x_1)^2
        \end{equation*}
        \begin{enumerate} [wide, labelwidth=!, labelindent=0pt,label=\textbf{\arabic*}.]
            \item $f$ có phải là hàm lồi không? Tại sao?

            Ta có $\mathrm{dom}(f)=\mathbb{R}^{2}$ là một tập lồi. Ta tính Gradient của $f(x)$ theo $x_1, x_2$:

            \begin{equation*}
                \nabla f(x) = \begin{bmatrix} \dfrac{\partial f(x)}{\partial x_1} \\ \dfrac{\partial f(x)}{\partial x_2}\end{bmatrix} = \begin{bmatrix} -400x_1(x_2 - x_1^2) -2(1-x_1) \\ 200(x_2 - x_1^2) \end{bmatrix}
            \end{equation*}

            Ta tính ma trận Hessian của $f(x)$ theo $x_1, x_2$:

            \begin{equation*}
                \nabla^2 f(x)=\begin{bmatrix} \dfrac{\partial^2 f(x)}{\partial x_1^2} & \dfrac{\partial^2 f(x)}{\partial x_1\partial x_2} \\ \dfrac{\partial^2 f(x)}{\partial x_2 \partial x_1 } & \dfrac{\partial^2 f(x)}{\partial x_2^2 } \end{bmatrix} = \begin{bmatrix} -400 x_2 + 1200x_1^2 + 2 & -400 x_1 \\ -400x_1 & 200 \end{bmatrix}
            \end{equation*}

            Ta nhận thấy $\nabla^2 f(x)$ tồn tại và liên tục với mọi $(x_1, x_2) \in \mathbb{R}^{2}$

            Ta xét:

            \begin{equation*}
                \det (\begin{bmatrix} -400 x_2 + 1200x_1^2 + 2 \end{bmatrix})=-400 x_2 + 1200x_1^2 + 2
            \end{equation*}

            Dấu của định thức này phụ thuộc vào $x_1, x_2$.
            \begin{itemize}
                \item Nếu $x_2 > \dfrac{1200x_1^2 + 2}{400}$ thì định thức $\det (\begin{bmatrix} -400 x_2 + 1200x_1^2 + 2 \end{bmatrix})<0$
                \item Nếu $x_2 < \dfrac{1200x_1^2 + 2}{400}$ thì $\det (\begin{bmatrix} -400 x_2 + 1200x_1^2 + 2 \end{bmatrix})>0$
            \end{itemize}
            Do định thức con chính đầu tiên lớn hơn 0 hay nhỏ hơn 0 phụ thuộc vào $x_1, x_2$ nên ta không cần xét thêm các định thức con chính tiếp theo. 
            Ma trận Hessian $\nabla^2 f(x)$ là ma trận không xác định dấu.

            $\nabla^2 f(x)$ là ma trận không xác định dấu nên $f(x)$ không là hàm lồi

            \item Hãy tìm tất cả các điểm cực tiểu địa phương của $f$. Hàm $f$ có điểm cực tiểu toàn cục không? Nếu có hãy chỉ ra điểm đó là điểm nào?
            
            Để tìm các điểm dừng, ta giải phương trình $\nabla f(x)=0$:

            \begin{equation*}
                \nabla f(x) = \begin{bmatrix} \dfrac{\partial f(x)}{\partial x_1} \\ \dfrac{\partial f(x)}{\partial x_2}\end{bmatrix} = \begin{bmatrix} -400x_1(x_2 - x_1^2) -2(1-x_1) \\ 200(x_2 - x_1^2) \end{bmatrix} = \begin{bmatrix} 0 \\ 0 \end{bmatrix}
            \end{equation*}

            \begin{equation*}
                \begin{aligned}
                    &\Leftrightarrow\begin{cases} -400x_1(x_2 - x_1^2) -2(1-x_1) = 0 \\ 200(x_2 - x_1^2) = 0 \end{cases} \\
                    &\Leftrightarrow \begin{cases} -400x_1(x_2 - x_1^2) -2(1-x_1) = 0 \\ x_2 = x_1^2 \end{cases} \\
                    &\Leftrightarrow \begin{cases} x_1=1 \\ x_2 = x_1^2=1 \end{cases} \\
                    &\Leftrightarrow \begin{cases} x_1=1 \\ x_2=1 \end{cases}
                \end{aligned}
            \end{equation*}
            Ta sẽ xét ma trận Hessian $\nabla^2 f(x)$ tại các điểm dừng nghiệm của phương trình $\nabla f(x) = 0$:

            \begin{equation*}
                \nabla^2 f(x_1=1, x_2=1)=\begin{bmatrix} 802 & -400 \\ -400 & 200 \end{bmatrix}
            \end{equation*}

            Ta xét:

            \begin{equation*}
                \begin{aligned}
                    &\det(\lbrack 802 \rbrack)=802>0 \\
                    &\det{\Big(\begin{bmatrix} 802 & -400 \\ -400 & 200 \end{bmatrix} \Big)}=802\times200 - (-400)\times(-400)=400>0 \\
                    \Rightarrow&\nabla^2 f(x_1=1, x_2=1) \succ 0
                \end{aligned}
            \end{equation*}

            Vì $\nabla^2 f(x)$ tồn tại và liên tục với mọi $(x_1, x_2) \in \mathbb{R}^{2}$ nên $\nabla^2 f(x, y)$ cũng tồn tại và liên tục trong một lân cận mở của $(x_1, x_2)=(1,1)$, $\nabla f(x_1=1, x_2=1)=0$ và $\nabla^2 f(x_1=1, x_2=1)$ là ma trận xác định dương nên $(x_1, x_2)=(1, 1)$ là một cực tiểu địa phương ngặt. Ta có $f(x_1=1, x_2=1)=0$. Mặt khác:

            \begin{equation*}
                \begin{aligned}
                    &f(x)=100(x_2 - x_1^2)^2 + (1-x_1)^2 \geq 0 = f(x_1=1,x_2=1) \thickspace \forall (x_1, x_2) \in \mathbb{R}^{2} \\
                    \Rightarrow&f(x_1=1,x_2=1) \leq f(x_1, x_2) \thickspace \forall (x_1, x_2) \in \mathbb{R}^{2}
                \end{aligned}
            \end{equation*}
            Vì vậy $(x_1, x_2)=(1, 1)$ cũng là một cực tiểu toàn cục của hàm $f(x)$.

            Kết luận: Hàm $f(x)$ chỉ có một cực tiểu địa phương $(x_1, x_2)=(1,1)$ và cũng chính là cực tiểu toàn cục của hàm $f$.
        \end{enumerate}
    \end{loigiai}

    \newpage
    \printbibliography[title={TÀI LIỆU THAM KHẢO}]
\end{document}